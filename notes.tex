\documentclass[12pt]{article}
\usepackage{graphicx} % Required for inserting images
\usepackage[margin=1in]{geometry}
\usepackage{fancyhdr}
\usepackage{graphicx}
\usepackage{enumitem}
\usepackage{amsmath}
\usepackage{amsfonts}
\usepackage{amsthm}
\everymath{\displaystyle}
\rfoot{\thepage}
\setlength{\headheight}{15pt}

\DeclareMathOperator{\obj}{Obj}
\DeclareMathOperator{\arw}{Arw}
\DeclareMathOperator{\dom}{Dom}
\DeclareMathOperator{\cod}{Cod}
\DeclareMathOperator{\id}{Id}

\theoremstyle{definition}
\newtheorem{definition}{Definition}

\theoremstyle{definition}
\newtheorem{theorem}{Theorem}

\theoremstyle{definition}
\newtheorem{lemma}{Lemma}

\theoremstyle{definition}
\newtheorem{axiom}{Axiom}

\begin{document}
\begin{flushleft}

\section{Sequences and Series}
\begin{definition}
    If $(a_n)_n$ is a sequence then $\lim a_n = L$ iff $\forall\varepsilon > 0$, $\exists N \in \mathbb{N}$ such that if $n \in \mathbb{N}$ where $n \geq N$ then $|a_n - L| < \varepsilon$
\end{definition}
\begin{definition}
    If $a \in \mathbb{R}$, then the $\varepsilon$-neighborhood of $a$ is $V_\varepsilon(a) = \{ x \in \mathbb{R} : |x - a| < \varepsilon \}$
\end{definition}
\begin{lemma}
    If $a_n = \frac{1}{\sqrt{n}}$, then $\lim a_n = 0$.
\end{lemma}
\begin{proof}
    Let $N \in \mathbb{N}$ such that $N > \frac{1}{\varepsilon^2}$, then
    \begin{align*}
        N > \frac{1}{\varepsilon^2} \iff \frac{1}{N} < \varepsilon^2 \iff \frac{1}{\sqrt{N}} < \varepsilon
    \end{align*}
    Now consider all $n \in \mathbb{N}$ where $n \geq N$, then
    \begin{align*}
        |a_n - 0| = \left|\frac{1}{\sqrt{n}}\right| = \frac{1}{\sqrt{n}} \leq \frac{1}{\sqrt{N}} < \varepsilon
    \end{align*}
\end{proof}
\begin{lemma}
    If $a_n = \frac{n + 1}{n}$, then $\lim a_n = 1$.
\end{lemma}
\begin{proof}
    Let $N \in \mathbb{N}$ such that $N > \frac{1}{\varepsilon}$, then
    \begin{align*}
        N > \frac{1}{\varepsilon} \iff \frac{1}{N} < \varepsilon
    \end{align*}
    Now consider all $n \in \mathbb{N}$ where $n \geq N$, then
    \begin{align*}
        |a_n - 1|
        = \left|\frac{n + 1}{n} - 1\right|
        = \left|\frac{n + 1}{n} - \frac{n}{n}\right|
        = \left|\frac{1}{n}\right|
        = \frac{1}{n}
        \leq \frac{1}{N}
        < \varepsilon
    \end{align*}
\end{proof}
\begin{lemma}
    If \[
        (a_n)_n = \left(1, -\frac{1}{2}, \frac{1}{3}, -\frac{1}{4}, \frac{1}{5}, -\frac{1}{5}, \frac{1}{5}, \ldots\right)
    \]
    then $(a_n)_n$ does not converge to $0$.
\end{lemma}
\begin{proof}
    Let $\varepsilon = \frac{1}{5}$, then consider all $N \in \mathbb{N}$ and consider $n = N$. Then notice that for $|a_n - L| = |a_n|$ must be greater or equal to than $\frac{1}{5}$, therefore $|a_n - L| \geq \frac{1}{5}$.
\end{proof}

\begin{definition}
    Let $x \mapsto \lfloor x \rfloor$ where $\lfloor x \rfloor \in \mathbb{Z}$ be the greatest value that satisfies $\lfloor x \rfloor \leq x$.
\end{definition}

\begin{lemma}
    If $a_n = \left\lfloor\frac{1}{n}\right\rfloor$, $n \geq 1$, then $\lim a_n = 0$
\end{lemma}
\begin{proof}
    Consider that if $n > 1$ then $\frac{1}{n} < 1$ so $\left\lfloor\frac{1}{n}\right\rfloor = 0$. \\
    For all $\varepsilon > 0$, let $N = 2$. Now consider all $n \in \mathbb{N}$ such that $n \geq N$ then $n > 1$ so therefore $\left\lfloor\frac{1}{n}\right\rfloor = 0$. Consider that
    \begin{align*}
        |a_n - L| = \left|\left\lfloor\frac{1}{n}\right\rfloor\right| = |0| < \varepsilon
    \end{align*}
\end{proof}

\begin{definition}
    A sequence $(a_n)_n$ converges to infinity iff $\forall M > 0$, $\exists N \in \mathbb{N}$ where $N > 0$ such that if $n \in \mathbb{N}$ where $n \geq N$ then $a_n > M$.
\end{definition}

\begin{lemma}
    If $a_n = \sqrt{n}$, then $(a_n)_n$ converges to infinity.
\end{lemma}
\begin{proof}
    Consider all $M > 0$ and let $N \in \mathbb{N}$ such that $N > M^2$ so $\sqrt{N} > M$.
    Now consider all $n \in \mathbb{N}$ such that $n \geq N$ so
    \begin{align*}
        a_n
        &= \sqrt{n}
        \geq \sqrt{N}
        > M
    \end{align*}
\end{proof}

\begin{definition}
    A sequence $(a_n)_n$ is bounded iff $\exists M > 0$ such that $\forall n \in \mathbb{N}$ that $|a_n| \leq M$
\end{definition}

\begin{theorem}
    If $\lim a_n = L$, then $(a_n)_n$ is bounded.
\end{theorem}
\begin{proof}
    Let $\varepsilon = 1$, then consider some $N \in \mathbb{N}$ that satisfies the convergence criterion for $\varepsilon$. Let $M = \left|\max(a_0, \ldots, a_{N - 1})\right| + 1 + |L| > 0$. Then trivially $|a_n| \leq M$ for $n \in \{0, \ldots, N - 1\}$. \\
    Now consider that $n \in \mathbb{N}$ such that $n \geq N$
    \begin{align*}
        |a_n|
        &= |a_n - L + L|
        \leq |a_n - L| + |L|
        < \varepsilon + |L|
        = 1 + |L| \leq M
    \end{align*}
\end{proof}

\end{flushleft}
\end{document}